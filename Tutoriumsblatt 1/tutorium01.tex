\documentclass[12pt, a4paper]{article}
\usepackage[left=1cm,right=1cm,top=2cm,bottom=2cm, bindingoffset=5mm]{geometry}

\usepackage{amsmath}
\usepackage{amssymb}
\usepackage{enumitem}
\usepackage[explicit]{titlesec}

\newcommand{\R}{\mathbb{R}}
\newcommand{\e}{\mathbf{e}}

\titleformat{\section}{\Large\bfseries}{}{0mm}{#1 \thesection}
\newcommand{\ex}{\section{Aufgabe}}
\renewcommand{\P}{\mathbb{P}}

\begin{document}
	\ex
		\begin{enumerate}[label=\alph*), labelwidth=!, labelindent=0pt]
			\item
			\item Zeigen Sie, dass
				\[
					(a_1, a_2, a_3) = \prod_{p\in\P} p^{\min\{\nu_p(a_1), \nu_p(a_2), \nu_p(a_3)\}}
				\]
				\textbf{Beweis:} Es gilt
				\begin{align*}
					a_1 &= \prod_{p\in\P} p^{\nu_p(a_1)} & a_2 &= \prod_{p\in\P} p^{\nu_p(a_2)} &a_3 &= \prod_{p\in\P} p^{\nu_p(a_3)} &
				\end{align*}
				Damit werden $a_1, a_2$ und $a_3$ höchstens $\min\{\nu_p(a_1), \nu_p(a_2), \nu_p(a_3)\}$-mal mit einer Primzahl $p \in \P$ geteilt, insbesondere gilt dann:
				\[
					p^{\min\{\nu_p(a_1), \nu_p(a_2), \nu_p(a_3)\}} \mid (a_1, a_2, a_3)
				\]
				Nun sind aber Primfaktorzerlegungen eindeutig und damit kann es keinen Teiler  von  $(a_1,a_2,a_3)$ geben, welcher sich nicht p-adisch darstellen lässt.
				\[
					\Rightarrow \qquad (a_1, a_2, a_3) = \prod_{p\in\P} p^{\min\{\nu_p(a_1), \nu_p(a_2), \nu_p(a_3)\}}
				\] \hfill $\square$
				\item induktiv können wir $(a_1, \dots, a_n)$ zu $(a_1, (a_2, \dots, a_n))$ umschreiben und so den ggT endlich vieler Zahlen $a_1,\dots,a_n \in \mathbb{Z}$ bestimmen. Ebenso folgt dann, dass
				\[
					(a_1, \dots, a_n) = \prod_{p\in\P}p^{\min\{\nu_p(a_1),\;\min\{\nu_p(a_2), \dots, \nu_p(a_n)\}\}} = \prod_{p\in\P}p^{\min\{\nu_p(a_1),\;\dots,\;\nu_p(a_n)\}}
				\] \hfill $\square$
				\item Es gilt $[a_1,a_2](a_1,a_2) = |a_1a_2|$\\
				\textbf{Beweis:} Wir betrachten
				\[
					|a_1a_2| = \prod_{p\in\P}p^{\nu_p(a_1)}\prod_{p\in\P}p^{\nu_p(a_2)} = \prod_{p\in\P}p^{\nu_p(a_1)}p^{\nu_p(a_2)} = \prod_{p\in\P}p^{\nu_p(a_1)+\nu_p(a_2)} \quad.
				\]
				Nun ist entweder $\nu_p(a_1) > \nu_p(a_2)$, $\nu_p(a_1) < \nu_p(a_2)$ oder $\nu_p(a_1) = \nu_p(a_2)$. Damit ist $\min$ und $\max$ eine der beiden Bewertungen $\nu_p(a_1)$ oder $\nu_p(a_2)$ und $\min = \max$ nur wenn $\nu_p(a_1) = \nu_p(a_2)$ gilt.
				\[
					\Rightarrow \qquad [a_1,a_2](a_1,a_2) = \prod_{p\in\P}p^{\min\{\nu_p(a_1),\nu_p(a_2)\}+\max\{\nu_p(a_1),\nu_p(a_2)\}} =  \prod_{p\in\P}p^{\nu_p(a_1)+\nu_p(a_2)} = |a_1a_2|
				\]  \hfill $\square$
		\end{enumerate}

\newpage

	\ex
		Wir betrachten die beiden diophantischen Gleichungen
		\[
			(i)\quad 561x_1 + 273x_2 + 117x_3 = 27 \qquad \text{und} \qquad (ii)\quad 561x_1 + 273x_2 + 117x_3 = 41\quad.
		\]
		\begin{enumerate}[label=\alph*), labelwidth=!, labelindent=0pt]
			\item Welche der beiden Gleichungen besitzt eine ganzzahlige Lösung $x_1, x_2, x_3$?\\
				\textbf{Beweis:} Wir betrachten die beiden Gleichungen im $\mathbb{Z}_3$:
				\[
					(i)\quad 0x_1 + 0x_2 + 0 = 0 \qquad \text{und} \qquad (ii) \quad  0x_1 + 0x_2 + 0x_3 = 2\quad.
				\]
				Damit lässt sich also nur $(i)$ ganzzahlig Lösen.
			\item z.B. $x_1 = 7$, $x_2 = 2$ und $x_3 = -38$
		\end{enumerate}

	\ex
		Wir betrachten die kanonische Quotientenabbildung $\pi : \mathbb{Z} \to \mathbb{Z}_4$ und schreiben $[x] := \pi(x)$  für die Restklasse von $x$ in $\mathbb{Z}_4$.
		\begin{enumerate}[label=\alph*), labelwidth=!, labelindent=0pt]
			\item Ist $p \in \P \setminus \{2\}$, so ist $[p] \in \{[1], [-1]\}$.\\
			\textbf{Beweis:} Widerspruch: Wäre $[p] \in \{[0], [2]\}$, dann würde $2 \mid p$ gelten, denn $2 \mid 4$, und $p$ wäre keine Primzahl. \hfill $\square$
			\item Ist $n \in \mathbb{N}$ mit $[n] = [-1]$, so existiert ein Primteiler $p$ von $n$ mit $[p] = [-1]$.\\
			\textbf{Beweis:} Es gilt
			\[
				n = \prod_{p\in\P} p^{\nu_p(n)} \Rightarrow [n] = \left[\prod_{p\in\P} p^{\nu_p(n)}\right] = \prod_{p\in\P} [p]^{\nu_p(n)}
			\]
			Da nach a) $p \in \{[-1], [1]\}$ gilt muss für $[n] = [-1]$ ein Faktor $[-1]$ existieren.\\
			Umgekehrt gilt die Aussage nicht: Bsp. $[-1]^2 = [1] = n \neq [-1]$ enthält $[-1]$ als Faktor. \hfill $\square$
			\item Es gibt unendlich viele $p\in\P$ mit $[p] = [-1]$.\\
			\textbf{Beweis:} Es sei $\P_n$ die Menge der ersten $n \in \mathbb{N}$ Primzahlen. Da $2 \in \P_1 \subsetneq \P_2 \subsetneq \dots \subsetneq\P_n$ ist und $\forall p \in \P \setminus \{2\} : [p] \in \{[-1], [1]\}$, und somit $[2][-1] = [-2] = [2]$, gilt:
				\[
					\left[1 + \prod_{p\in\P_n}p\right] = [1] + \left[\prod_{p\in\P_n}p\right] = [1] + [2]\left[\prod_{p\in\P_n\setminus\{2\}}p\right] = [1] + [2][\pm1] = [-1]
				\]
				Da alle Faktoren in $m := 1 + \prod_{p\in\P_n}p$ teilerfremd zu allen $\P_n$ sind, aber $[m] = [-1]$ existiert nach b) ein teilerfremder Primfaktor $p$ von $m$ mit $[p] = [-1]$. Wähle $n$ beliebig groß $\Rightarrow$ Behauptung \hfill $\square$
		\end{enumerate}
\end{document}